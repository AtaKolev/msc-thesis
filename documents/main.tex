\documentclass{article}
\usepackage{graphicx} % Required for inserting images
\usepackage[T1, T2A]{fontenc}% T2A for Cyrillic font encoding
\usepackage[bulgarian]{babel}

\title{Прилагане на методи от машинното обучение за прогнозиране и анализ на състоянието на търкалящи лагери}
\author{Атанас Николаев Колев}
\date{06/06/2024}
\begin{document}


\maketitle
% Insert the image
\begin{figure}[h] % h stands for 'here'
    \centering
    \includegraphics[width=1\textwidth]{./photos/NBU.png} % Replace with your image file name
    \label{fig:example-image}
\end{figure}
\newpage
\tableofcontents
\newpage
\part{Увод}
В днешно време машинното обучение намира широко приложение в различни индустриални сектори, като осигурява нови методи за диагностика и прогноза на техническото състояние на различни механични системи. Един от критичните компоненти в много машини и оборудване са търкалящите лагери. Тяхната надеждност и безаварийна работа са от съществено значение за предотвратяване на производствени загуби и поддържане на висока производителност. Настоящата дипломна работа ще разгледа методи за обработка и анализ на данни за търкалящи лагери чрез алгоритмите за машинно обучение, с цел подобряване на ефективността на превантивната диагностика.
\section{Цел на дипломната работа}
\paragraph{Целите на настоящата дипломна работа включват:}
\begin{itemize}
    \item Изследване, разработване и внедряване на модел за прогнозиране и анализ на състоянието на търкалящи лагери като се използват алгоритми за машинно обучение;
    \item Разработване на уеб-базирано приложение, базирано на получения модел, за предоставяне на информация в реално време за състоянието на лагерите.
\end{itemize}
\section{Задачи}
\paragraph{За постигане на целите на дипломната работа са поставени следните задачи:}
\begin{itemize}
    \item Запознаване с устройството, принципа на работа, експлоатационните характеристики и проблемите на търкалящите лагери.
    \item Анализ на възможностите за получаване на данни за търкалящи лагери, тяхната структура и характеристики. Предварителната обработка и подготовка на данните за използване в моделите за машинно обучение.
    \item Запознаване с възможностите на алгоритмите за машинно обучение за откриване на аномалии в данните и разграничаване на типове повреди. В това число се предвижда проучване на възможностите за приложения на хармоничен и Cepstrum анализ.
    \item Разработване на реален модел за откриване на аномалии и класификация на повреди, трениран с реални данни. Тестване и валидиране на модела. Внедряване на модела за работа в реално време и мониторинг на неговото изпълнение.
    \item Разработка на уеб-базирано приложение, реализиращо разработения модел. За целта се предвижда използване на контейнеризация чрез Docker и внедряване на приложението на локален сървър.
\end{itemize}
\section{Резултати}
Резултатите от дипломната работа са свързани с матемитечското моделиране и създаването на алгоритъм за прогнозиране и анализ на състоянието на търкалящи лагери. Предвижда се изграждането на уеб-базирано приложение, което да предоставя информация за състоянието на лагерите и да подпомага вземането на решение по отношение на тяхната навременна поддръжка. Чрез идентифицирането и класифицирането на типовите повреди се цели реализация на превантивна диагностика, с чиято помощ аварийният ремонт се трансформира в планов.
\newpage
\part{Запознаване с търкалящи лагери и алгоритмите за машинно обучение}
\setcounter{section}{0}
\section{Лагери - принцип на работа и повреди}
\subsection{Принцип на работа}
\subsection{Повреди}
\section{Машинно обучение - теория и приложения}
\subsection{Теоретична постановка}
\subsection{Приложения на машинното обучение}
\newpage
\part{Резултати}
\setcounter{section}{0}
\section{Обработка и анализ на данните}
\section{Модел}
\section{Уеб-базирано приложение}
\newpage
\part{Заключение}
\setcounter{section}{0}
\newpage
\part{Приложения}
\setcounter{section}{0}
\section{GitHub репозитория за програмния код на решението}
\section{Фигури}
\newpage
\part{Източници}
\setcounter{section}{0}
\section{Книги}
\section{Статии}
\end{document}
